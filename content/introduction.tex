\section{Introduction} \label{sec:introduction}

% Introduction to the topic, why are datacenters important?
Datacenter infrastructure is a fundamental building block of our society, undergoing a digitization process.
Digital data has been the driving force behind many of the recent technological advancements.
It is expected that by 2025, 463 exabytes of data will be generated daily.
The impact of such growth is reflected in the economic sector, where every \euro 1 invested in the ICT infrastructure generates a notably higher added value of \euro 15~\cite{iosup2022future}.
Stakeholders in detrimental areas of the world, such as government entities, educational institutions and economic organization rely on cloud services to maintain their operations.
Even though cloud adoption rates are increasingly high, stakeholders are requiring performance and reliability guarantees from such systems, while also expecting cost-effective solutions.

% State the problem: it is particularly challenging to design a datacenter that is both performant and cost-efficient.
Designing and operating a datacenter that is both performant and cost-efficient is a challenging task.
On one hand, it is essential to manage the trade-off between performance and cost in a way that enables scalability and reliability.
On the other hand, energy consumption and the environmental impact of datacenters need to be minimized to ensure sustainability.

% Datacenter simulation, a (possible and effective) solution to the problem.
The complex problem of designing and operating a datacenter has been proven to be effectively addressed through simulation~\cite{DBLP:conf/ccgrid/MastenbroekAI+21}.
Datacenter simulation tools have been created with the sole purpose of providing a reliable way to predict certain parameters of a datacenter's performance, that enable stakeholders to make informed decisions.
However, as of right now, simulation tools have not reached a maturity level that would allow them to be used by non-experts in the field of datacenter design and operation.
As the industry rapidly evolves, so do the requirements that have to be met by a reliable simulation tools.
Therefore, most researchers and developers of such software prefer to focus on the core functionality of the tool, rather than on the user experience.

% Purpose of the work
We have established that datacenter simulation is a necessary method in the process of enhancing the performance and efficiency of datacenters. %>> 'method' here can be amibiguous
We analyze through this work the potential of an ICT digital twin as a service, as well as the implications of designing and assessing the usability of such a service.